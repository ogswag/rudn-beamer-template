\documentclass[10pt, hyperref={bookmarks=true}, aspectratio=169]{beamer}
\hypersetup{bookmarksnumbered=true}
\usepackage[utf8]{luainputenc}
\usepackage{polyglossia}
\setdefaultlanguage{russian}
\setotherlanguage{english}

\usetheme[minimal]{RUDN}

%\setbeamertemplate{footline}[frame number]

%%%%%%%%%%%%%%%%%%%%%%%%%%%%%%%%%%%%%%%%%%%%%%%%%%%%%%%%%%%%%%%%%%%%%%%%%%%%%%%%
%% Настройки титульного слайда
%%%%%%%%%%%%%%%%%%%%%%%%%%%%%%%%%%%%%%%%%%%%%%%%%%%%%%%%%%%%%%%%%%%%%%%%%%%%%%%%

\title[Стиль презентации для РУДН]{\textbf{Стиль презентации для РУДН}}

\subtitle[Подзаголовок]{Подзаголовок}

\author[Александр Захаров]{Александр Захаров\inst{\tiny 1} \and Ещё автор\inst{\tiny 2}}

\institute[РУДН]{
    \inst{1}
    Математический институт им. Никольского\\
    ФФМиЕН, РУДН.
    \and
    \inst{2}
    Смехатический хохмститут им. Никулина\\
    КеК, ЛОЛ.
}

\date[МКК, 2025-04-01]{Международная конференция конференций\\ 1 Апреля, 2025гг.}

%%%%%%%%%%%%%%%%%%%%%%%%%%%%%%%%%%%%%%%%%%%%%%%%%%%%%%%%%%%%%%%%%%%%%%%%%%%%%%%%

%% Уберите или добавьте комментарии к строкам ниже, если хотите включить/выключить
%% повторный показ слайда с содержанием перед каждой новой главой презентации.

\AtBeginSection[]
{
    \begin{frame}
        \frametitle{Содержание - текущая секция}
        \tableofcontents[currentsection]
    \end{frame}
}

%%%%%%%%%%%%%%%%%%%%%%%%%%%%%%%%%%%%%%%%%%%%%%%%%%%%%%%%%%%%%%%%%%%%%%%%%%%%%%%%

\begin{document}

\begin{frame}[plain]
    \titlepage
\end{frame}

%%%%%%%%%%%%%%%%%%%%%%%%%%%%%%%%%%%%%%%%%%%%%%%%%%%%%%%%%%%%%%%%%%%%%%%%%%%%%%%%

%% Эта строка делает так, что номер слайда после титульного становится равен 1
\addtocounter{framenumber}{-1}
%% Если вам это не нужно, закомментируйте её

\begin{frame}
    \frametitle{Содержание}
    \tableofcontents
\end{frame}

%%%%%%%%%%%%%%%%%%%%%%%%%%%%%%%%%%%%%%%%%%%%%%%%%%%%%%%%%%%%%%%%%%%%%%%%%%%%%%%%

\section{Как работать с появлением текста на слайде}

%%%%%%%%%%%%%%%%%%%%%%%%%%%%%%%%%%%%%%%%%%%%%%%%%%%%%%%%%%%%%%%%%%%%%%%%%%%%%%%%

\begin{frame}
    \frametitle{Пример обычного слайда с поэтапным списком}

    Мы вынуждены отталкиваться от того, что убеждённость некоторых оппонентов требует от нас анализа позиций, занимаемых участниками в отношении поставленных задач.

    \begin{itemize}
        \item<1-> Список 1
        \item<2-> Список 2
        \item<3-> Список 3
    \end{itemize}

\end{frame}

%%%%%%%%%%%%%%%%%%%%%%%%%%%%%%%%%%%%%%%%%%%%%%%%%%%%%%%%%%%%%%%%%%%%%%%%%%%%%%%%

\begin{frame}
    \frametitle{Как использовать команду \texttt{\textbackslash{pause}}}
    Например, я ввёл этот текст и добавил \texttt{\textbackslash pause}, \pause

    теперь этот текст и формулы появятся только на следующем слайде
    \begin{align*}
        f(x) & = x^2         \\
        g(x) & = \frac{1}{x} \\
        F(x) & = \int^a_b \frac{1}{3}x^3
    \end{align*}

    \pause

    а этот текст и того позднее.
\end{frame}

%%%%%%%%%%%%%%%%%%%%%%%%%%%%%%%%%%%%%%%%%%%%%%%%%%%%%%%%%%%%%%%%%%%%%%%%%%%%%%%%

\section{Выделение текста}

%%%%%%%%%%%%%%%%%%%%%%%%%%%%%%%%%%%%%%%%%%%%%%%%%%%%%%%%%%%%%%%%%%%%%%%%%%%%%%%%

\begin{frame}
    \frametitle{Выделение текста}

    Вот так можно обратить внимание слушателей \alert{на текст командой \texttt{\textbackslash alert}}.

    \begin{block}{Люблю грозу в начале мая,}
        Когда весенний, первый гром,
    \end{block}

    \begin{alertblock}{Как бы резвяся}
        и играя,
    \end{alertblock}

    \begin{examples}
        Грохочет в небе голубом.
    \end{examples}

\end{frame}

%%%%%%%%%%%%%%%%%%%%%%%%%%%%%%%%%%%%%%%%%%%%%%%%%%%%%%%%%%%%%%%%%%%%%%%%%%%%%%%%

\begin{frame}
    \frametitle{Текст в две колонки с картинкой}

    \begin{columns}

        \column{0.45\textwidth}

        \begin{figure}
            \centering
            \includegraphics[width=\columnwidth]{./img/manuls.jpg}
            \caption{{Котята манулов} \newline \tiny{(Загружено c https://ria.ru/20201208/manul-1588107027.html)}}
            \label{картинка:котята-манулов}
        \end{figure}

        \column{0.55\textwidth}
        А это текст во второй колонке!

        \texttt{\textbackslash (0o0)/}

    \end{columns}
\end{frame}

%%%%%%%%%%%%%%%%%%%%%%%%%%%%%%%%%%%%%%%%%%%%%%%%%%%%%%%%%%%%%%%%%%%%%%%%%%%%%%%%

\end{document}
